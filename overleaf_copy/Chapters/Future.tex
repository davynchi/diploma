% Chapter Template

\chapter{Заключение} % Main chapter title

\label{Future} % Change X to a consecutive number; for referencing this chapter elsewhere, use \ref{ChapterX}

%----------------------------------------------------------------------------------------
%	SECTION 1
%----------------------------------------------------------------------------------------

\section{Выводы}

В этой работе была создана теория для работы стратегий в задаче о многоруких бандитах с учетом неприятия к риску. Был создан алгоритм StandardGreedy, при заданных матожиданиях и дисперсиях вычислияющий оптимум функции полезности на симплексе. Были адаптированы стратегии из классической задачи о многоруких бандитах для измененной задачи, предложена стратегия с коррекцией дисперсии. Стандартные стратегии были протестированы в классической задаче для распределений, отличных от нормального. В результате получено снижение эффективности стандартных алгоритмов для распределений Стьюдента с малым числом степеней свободы и низкую эффективность известных стратегий для распределения Коши. Адаптированные стратегии были протестированы для измененной задачи и распределений Стьюдента с различным числом степеней свободы. Получена высокая эффективность алгоритмов при $\nu \geq 3$ и низкая при $\nu = 2.1$ и $\nu = 2$. Показано, что проблемы носят фундаментальный характер, поэтому сделан вывод, что оценка риска через дисперсию при $\nu$, близких к 2, слабоприменима.


\section{Направления развития}

В этой работе было проведено всесторонне исследование задачи о многоруких бандитах с учетом степени отвращения к риску. Однако есть моменты, которые можно прояснить и исследовать глубже в будущих работах:
\begin{enumerate}
    \item Сэмплирование Томпсона представляет собой эффективный алгоритм для нахождения оптимального рычага в классической задаче о многоруких бандитах. В этой работе была предложена адаптация этой стратегии для измененной задачи, но не было проведено исследования этой адаптации.
    \item Был рассмотрен только один вариант параметров $\delta$ и $\tau$ в VDBE. Настройка гиперпараметров способна значительно улучшить эффективность стратегии.
    \item Некоторые из загадочных явлений ввиду ограниченности бакалаврской работы не были объяснены. Например, почему при уменьшении числа степеней свободы максимальный процент оптимальных действий для жадной позитивной инициализации с const step-size увеличивается? Почему для этой де стратегии наблюдается эффект переобучения?
\end{enumerate}
Эти проблемы могут быть решены в ходе создания магистерского диплома.